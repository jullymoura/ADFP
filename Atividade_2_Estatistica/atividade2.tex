\documentclass{article}
\usepackage{graphicx} % Required for inserting images

\begin{document}

 % Capa
    \begin{titlepage}
      \centering
      \vspace*{1cm}
    
     \Huge
     \textbf{Atividade 2 - Introdução à análise de dados de Física de Partículas}

     \vspace{0.5cm}
    
     \Large 
      Jully Moura Alves

     \vfill
    
     \Large
      Outubro de 2024
    \end{titlepage}

 %Pagina 1
    \raggedright{
      \section*{Exercício 1}
      \vspace*{1cm}

 
      \textbf{A equação de ajuste linear é dada por:}
      \vspace{0.5cm}

       %Eq Ajuste linear 
           {\Large 
              \[ y = mx + b\]
              
              \vspace{0.5cm}
              
              \[\Delta y = \frac {\sqrt {\sum_{i=1}^N (m x_i + b - y_i)^2}}{N-2} \]

              \vspace{0.5cm}
              
              \[ y \pm \Delta y = (m \pm \Delta m) x + (b \pm \Delta b) \]
           }
  
       \vspace{0.5cm}

       %Eq m
           \textbf{m:}
           \vspace{0.5cm}
  
           {\Large
              \[ m = \frac{\sum_{i=1}^{N} x_i y_i -({\sum_{i=1}^{N} x_i \sum_{i=1}^{N} y_i})\frac{1}{N}}{ \sum_{i=1}^{N} x_i^2 - \frac{1}{N}{\left( \sum_{i=1}^{N} x_i \right)^2}}\]
              
            \vspace{0.5cm}
  
              \[\Delta m = \frac{\Delta y}{\sqrt{ (x_i - \frac{ \sum_{i=1}^N x_i}{N})^2}}\]
            }
  
        \vspace{0.5cm}
  
        %Eq b
            \textbf{b:}
            \vspace{0.2cm}
  
            {\Large
               \[ b = \frac{1}{N}({\sum_{i=1}^{N} y_i - m \sum_{i=1}^{N} x_i)}\] 
             
             \vspace{0.2 cm}
          
               \[\Delta b = \sqrt{\frac{\sum_{i=1}^Nx_i^2}{N \sum_{i=1}^N (x_i - \frac{\sum_{i=1}^N x_i}{N})^2}} \Delta y\]

                             
            }
%Página 2
    \newpage
       {\section*{Exercício 2}}
       {\Large 
          {\sigma: }\\
          {\[ \sigma = \frac{N_{total} - N_{background}}{L}\]
          
           \vspace{0.2cm}
           
           \[ \sigma = \frac {2467 - 1223.5}{25} = 53.74 (fb) \]

           \vspace{1cm}

           \textbf{N:}
           \[ N = N_{total} - N_{background} \]
           \vspace{0.2 cm}
           \[ N = 2467 - 1223.5 = 1343.5\]

           \vspace{1cm}
           
           \textbf{Incertezas estatísticas}

           \vspace{1cm}

           \textbf{\Delta N:}\\
           \[ \Delta N = \sqrt{N} \]
           \vspace{0.2 cm}
           \[ \Delta N = \sqrt{1343.5} = 36.65 \]
           
           \vspace{1cm}
           
           \textbf{\Delta \sigma_1:}\\   
           \[ \Delta \sigma_1 = \frac{ \Delta N}{L}\]
           \vspace{0.2 cm}
           \[ \Delta \sigma_1 = \frac{36,65}{25} = 1.47 (fb)\]
       
           \vspace{1cm}

           \textbf{Incertezas sistemáticas}

           \vspace{1cm}
           
           \textbf{\Delta L:}\\
           \[ \Delta L = 0.1 \times 25 = 2.5(fb^{-1})\]

            \vspace{1cm}

           \textbf{\Delta \sigma_2:}\\
           \[ \Delta \sigma_2 = \sigma \times \frac{ \Delta L }{L}\]
           \vspace{0.2 cm}           
           \[ \Delta \sigma_2 = 53.74 \times \frac{2.5}{25} = 5.37 (fb) \]
           
           \vspace{1cm}

           \textbf{Incerteza total}

           \vspace{1cm}

           \[\Delta \sigma = \sqrt{(\sigma_1)^2 + (\sigma_2)^2}\]
           \vspace{0.2 cm} 
           \[\Delta \sigma = \sqrt{(1.47)^2 + (5.37)^2} = 5.57(fb) \]

           \vspace{1cm}

           \textbf{Resultado}

           \vspace{1cm}

           \[ \sigma = (53.74 \pm 5.57)fb\]         
          }
       }  

 %Exercício 3   
    \newpage
      \section*{Exercício 3}

      \vspace{0.5cm}
      
      Através de Poisson

      \vspace{0.5cm}
      
     {\Large 
        \[P( k; \lambda) = \frac{\lambda^k e^{- \lambda} }{k!} \]
        
      \vspace{0.5cm}

        Para k = 0:

      \vspace{0.5cm}
         
       \[P( 0; \lambda) =  e^{- \lambda} \] 

      \vspace{0.5cm}

        Para intervalo de confiança de 95\% :

      \vspace{0.5cm}

        \[e^{- \lambda} \ge 1 - 0.95\]
        \vspace{0.2cm}
        \[e^{- \lambda} \ge 0.05}\]
        \vspace{0.2cm}
        \[ - \lambda \ge ln(0.05)\]
        \vspace{0.2cm}
        \[ \lambda \le - ln(0.05)\] 
        \vspace{0.2cm}
        \[ - ln(0.05) \approx 2.996\]


        Resultado:

      \vspace{0.5cm}
         
        \[ \lambda \ge 2.996 \]
      }

%Exercício 4
    \newpage

    \section*{Exercício 4}

    {\Large 
       \[ X^2 = \sum_{i=1}^N \frac {(y_i  - f(x_i))²}{ {\sigma_i}^2 } \]

       \vspace{0.5cm}

       Graus de liberdade:

       \vspace{0.5cm}

       \[ndf = N - p\]

       \vspace{0.5cm}

       Média de X²:

       \vspace{0.5cm}
       
       \[Med|X^2| = ndf\]

       \vspace{0.5cm}

       Variância de X²:

       \vspace{0.5cm}

       \[Var(X²) = 2ndf\]

       \vspace{0.5cm}

       Ajuste:

       \vspace{0.5cm}

       \[Mdf \left[ \frac{X^2}{ndf} \right] = \frac{Med|X^2|}{ndf}\]

       \vspace{0.5cm}

       \[\frac{Med|X^2|}{ndf} = \frac{ndf}{ndf}=1\]

       \vspace{9cm}

       Resposta:

       \vspace{1cm}

       Quando \[ \frac{X^2}{ndf} = 1\]

       \vspace{1cm}
       
       a discrepância entre os dados observados e os valores ajustados é consistente com a incerteza dos dados.

       \vspace{1cm}

        Quando \[ \frac{X^2}{ndf} `< 1\] 
        
        \vspace{1cm}
        
        a discrepância entre os dados e o modelo é menor do que o esperado, indicando que os erros podem estar superestimados ou o modelo é mais complexo do que os dados exigem.

        \vspace{1cm}

        Quando \[ \frac{X^2}{ndf} `< 1\] 
        
        \vspace{1cm}
        
        a discrepância é maior do que a incerteza permite, indicando que o modelo pode não estar adequado ou os erros subestimados.

        \vspace{2cm}

        Por esses motivos, a melhor função que se adequada aos dados é quando\vspace{1cm}
        
        \vspace{0.5cm}
        
        \[ \frac{X^2}{ndf} = 1\]

        
    }
    
  



\end{document}

